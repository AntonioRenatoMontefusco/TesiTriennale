\documentclass[12pt]{book}

\begin{document}
	\chapter*{Intruduzione}	
		Il CoronaVirus-19, abbreviato CoViD-19, è un virus trovato per la prima volta ad Wuhan, in Cina. Esso è conosciuto anche come \emph{Malattia respiratoria acuta}, abbreviato il \textbf{SARS-CoV-2}, dato il fatto che un virus che come principale metodo di trasmissione utilizza le vie aeree ha avuto vita facile nel diffondersi velocemente, in effetti tutti respiriamo. In breve tempo si è passati dall'isolamento della Cina al doverlo definire \textbf{pandemia}.
		Per la sua facilità di diffusione e per la pericolosità per alcuni soggetti più deboli abbiamo dovuto cambiare radicalmente le nostre vite, iniziando con i lockdown, dovendo passare alla didattica a distanza perdendo la socialità dei luoghi di studio, abbiamo visto l'economia rallentare e la socialità in presenza scomparire per via della chiusura temporanea di aziende definite "non di prima necessità".
		Questa situazione ha dato una motivazione per effettuare studi e ricerche in questo ambito, sia per ridurre i sintomi del virus, in buona parte già sta avvenendo con i vari vaccini, sia per permettere alle persone di incontrarsi e lavorare come prima della pandemia. In Italia stiamo allentando la presa sulle restrizioni grazie anche all'avvento del \emph{greenpass} il quale permette di certificare che si è vaccinati o che si hanno gli anticorpi dovuti ad un'infezione da CoViD19. Questo però non ferma il contagio dato che con il vaccino ti permette di avere sintomi più lievi ma è comunque possibile essere contagiati e diffondere il virus. 
		Uno degli aspetti che coinvolge un po' tutte le persone, sia per divertimento che per lavoro è il trasporto pubblico. Non tutti hanno un mezzo di trasporto proprio o per molti addirittura non conviene, basti pensare alle grandi città con lunghe code per il traffico a tal punto da fare prima a far il tragitto a piedi. Quindi molti optano per i trasporti pubblici ma qui è dove avvengono i principali contagi dovuti ai luoghi chiusi, stretti e poco arieggiati che impediscono di seguire perfettamente le normative dettate dal governo e dall'Organizzazione Mondiale della Sanità (OMS); questo è uno dei principali motivi del lavoro di tesi.\\
		L'obbiettivo del progetto di tesi è quello di creare una simulazione ad agenti estendibile a diversi tipi di percorsi e abitudini cittadine diverse, da cui recuperare dati sulle persone e sui contagi che avvengono su un mezzo di trasporto pubblico (nel nostro caso si tratta di un autobus). La simulazione è stata realizzata interamente in Unity3D integrando tool ed algoritmi per i vari problemi da risolvere. I dati sono stati presi da un articolo riguardante uno studio pre-pandemia sui pendolari della città giapponese di Obuse, della quale è stato preso il percorso della tratta del bus.
		\\
		La tesi sarà suddivisa nelle seguenti sezioni:
		\begin{itemize}
			\item \textbf{Stato dell'arte}: mostrerò lo stato delle ricerche sulla diffusione del CoViD-19 in ambito di trasporti pubblici, citando elaborati e mostrandone i punti di forza e le limitatezze.
			\item \textbf{Presentazione delle tecnologie}: in questa sezione mostrerò in dettaglio le tecnologie utilizzate per la creazione della simulazione.
			\item\textbf{Simulazione nel dettaglio}: mostrerò il lavoro svolto per la realizzazione della simulazione mostrando in dettaglio i vari oggetti, come sono stati utilizzati i tool e il lavoro svolto senza tool.
			\item\textbf{Risultati ottenuti}: mostrerò i risultati ottenuti da vari test della simulazione.
			\item\textbf{Conclusione e sviluppi futuri}: si analizzeranno le limitazioni del lavoro allo stato attuale e si metteranno in risalto idee per sviluppi futuri
		\end{itemize}
\end{document}