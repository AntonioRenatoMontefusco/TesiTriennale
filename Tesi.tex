\documentclass[12pt, openany]{book}
\usepackage[italian]{babel}
\usepackage{xcolor}
\newcommand\myworries[1]{\textcolor{red}{#1}}
\begin{document}
	\chapter*{Intruduzione}	
		Il CoronaVirus-19, abbreviato CoViD-19, è un virus trovato per la prima volta a Wuhan, in Cina. Esso è conosciuto anche come \emph{Malattia respiratoria acuta}, abbreviato in \textbf{SARS-CoV-2}, dato che è un virus che come principale metodo di trasmissione utilizza le vie aeree ha avuto vita facile nel diffondersi velocemente, in effetti tutti respiriamo. In breve tempo si è passati dall'isolamento della Cina al doverlo definire \textbf{pandemia} e quindi passare ad un isolamento globale.
		Per la sua facilità di diffusione e per la sua pericolosità per alcuni soggetti più deboli è stato necessario modificare lo stile di vita a cui si era abituati, iniziando con i lockdown, dovendo passare alla didattica a distanza perdendo la socialità dei luoghi di studio, abbiamo visto l'economia rallentare, nei casi peggiori crollare e la socialità in presenza scomparire per via della chiusura temporanea di aziende e luoghi definiti "non di prima necessità".
		Questa situazione ha dato una motivazione per effettuare studi e ricerche in questo ambito, sia per ridurre i sintomi del virus, in buona parte già sta avvenendo con i vari vaccini, sia per permettere alle persone di incontrarsi e lavorare come prima della pandemia trovando metodi per ridurre i contagi come le mascherine. In Italia stiamo allentando la presa sulle restrizioni grazie anche all'avvento del \emph{GreenPass} il quale permette di certificare che si è vaccinati o che si hanno gli anticorpi dovuti ad un'infezione da CoViD-19. Questo però non ferma il contagio dato che il vaccino permette agli individui di avere sintomi più lievi ma è comunque possibile essere contagiati e diffondere il virus. 
		Uno degli aspetti che coinvolge un po' tutte le persone, sia per divertimento che per lavoro è il trasporto pubblico. Non tutti hanno un mezzo di trasporto proprio o per molti addirittura non conviene, basti pensare alle grandi città con lunghe code per il traffico a tal punto da fare prima a far il tragitto a piedi. Quindi molti optano per i trasporti pubblici ma qui è dove avvengono i principali contagi dovuti ai luoghi chiusi, stretti e poco arieggiati che impediscono di seguire perfettamente le normative dettate dal governo e dall'Organizzazione Mondiale della Sanità (OMS); questo è uno dei principali motivi del lavoro di tesi.\\
		L'obbiettivo del progetto di tesi è quello di creare una simulazione ad agenti estendibile a diversi tipi di percorsi e abitudini cittadine diverse, da cui recuperare dati sulle persone e sui contagi che avvengono su un mezzo di trasporto pubblico (nel nostro caso si tratta di un autobus). La simulazione è stata realizzata interamente in Unity3D,un motore grafico di che consente lo sviluppo di contenuti interattivi digitali, sia 2D che 3D, come ad esempio videogiochi e simulazioni. Sono stati anche utilizzati tool ed algoritmi per risolvere  vari problemi , come ad esempio abbiamo utilizzato \emph{A* Pathfinding Project} a cura di Aron Granberg per il calcolo del percorso degli agenti. 
		\\Gli agenti vengono distinti in:
		\begin{itemize}
			\item Sano, indicato in verde
			\item Contagioso, indicato in rosso
			\item Infetto, indicato in giallo
			
		\end{itemize} 
	\myworries{Inserire immagine agenti}\\
		I dati sul numero di pendolari della tratta sono stati presi da un articolo riguardante uno studio pre-pandemia sui viaggiatori della cittadina giapponese di Obuse, della quale è stato preso il percorso della tratta del bus. La simulazione permette di selezionare la percentuale di contagiati iniziale e la percentuale di infezione del virus in modo tale da effettuare facilmente diversi test. Il numero di persone alle fermate è selezionabile per ogni singola fermata in modo tale da poter essere utilizzate in diversi percorsi o più semplicemente applicarlo alle stime dei pendolari attuali.\\
		All'avvio della simulazione viene generato un file .csv che verrà riempito per ogni tratta del bus con le informazioni generate sugli agenti, come il numero i agenti totali, quelli contagiosi, quelli infetti e quelli sani.
		\\
		La tesi sarà suddivisa nelle seguenti sezioni:
		\begin{itemize}
			\item \textbf{Stato dell'arte}: verrà mostrato lo stato delle ricerche sulla diffusione del CoViD-19 in ambito di trasporti pubblici, citando elaborati e mostrandone i punti di forza e le limitatezze.
			\item \textbf{Presentazione delle tecnologie}: in questa sezione verrà mostrato in dettaglio le tecnologie utilizzate per la creazione della simulazione.
			\item\textbf{Simulazione nel dettaglio}: verrà mostrato il lavoro svolto per la realizzazione della simulazione mostrando in dettaglio i vari oggetti, come sono stati utilizzati i tool e il lavoro svolto senza tool.
			\item\textbf{Risultati ottenuti}: verrà mostrato i risultati ottenuti da vari test della simulazione.
			\item\textbf{Conclusione e sviluppi futuri}: si analizzeranno le limitazioni del lavoro allo stato attuale e si metteranno in risalto idee per sviluppi futuri
		\end{itemize}
	\chapter{Stato dell'arte}
		Dalla scoperta del salto di specie effettuato dal CoViD-19 sull'essere umano ad oggi ci sono stati innumerevoli studi, ricerche su vari aspetti della malattia, sul suo impatto sulla società in ambito politico,sociale ed economico. Ci sono anche varie simulazioni sulla sua diffusione in varie situazioni come trasporti pubblici, uffici, ospedali, locali pubblici. \\
		Come primo articolo si parlerà di "Simulaion-based Estimation of the Spread of COVID-19 in Iran" pubblicato il 27 Marzo 2020 su medRxiv. Come primo luogo l'articolo mette in luce i problemi dovuti ai dati provenienti dai vari paesi colpiti dal virus, definendoli altamente inaffidabili perché non tengono conto che i sintomi lievi del CoViD-19 sono assimilabili al raffreddore stagionale o alla comune influenza, in più molti testi vengono effettuati tramite screening limitati, un metodo per effettuare esami a tappeto allo scopo di individuare una malattia. Tutti questi dati non vengono combinati con le statistiche ufficiali sulla diffusione del virus e si va quindi a sottovalutare la quantità di infetti.
		Lo studio è stato effettuato sulla diffusione del virus in Iran nei primi mesi del 2020.\\
		È stato sviluppato un modello dinamico per fornire un quadro affidabile dello stato della malattia basandosi sui dati esistenti. Il modello si basa sul framework \textbf{SEIR}(Suscettibile, Esposto, Infetto, Ricoverato); i modelli matematici, in epidemiologia, sono modelli simbolici costituiti da una o più equazioni che considerano diversi parametri per prevedere l'andamento di una malattia in diverse condizioni ambientali o per calcolare il rischio di morte o l'aspettativa di vita nel corso di una pandemia di specifiche entità. SEIR è uno di questi modelli matematici, esso considera il periodo di incubazione durante il quale un individuo è infetto ma non contagioso, in questo caso si trova nello stato di \emph{"Esposto"}, questo è il principale motivo per cui è stato scelto questo modello, tenendo conto il periodo di incubazione del CoVid-19 che ha un massimo di 14 giorni.\\
		Come primo parametro è stato registrata la frequenza di contatti medi tra persone man mano che venivano segnalati i decessi, poi sono stati differenziati i casi segnalati dai casi effettivi e di come questa differenza cambia con l'andare dell'epidemia(all'epoca ancora definibile epidemia). Nel modello sono stati utilizzati sia dati ufficiali che dati stimati dalla comunità medica in modo tale da ricostruire un quadro più completo della situazione epidemica. Questi dati vengono raccolti per poi passare ad un simulazione con metodo \emph{Monte Carlo}; è una classe di metodi computazionali basati sul campionamento casuale per ottenere risultati numerici, questo metodo è utilizzato per trarre stime attraverso simulazioni.
		Da quest'analisi sono state trovate sei possibili situazioni della diffusione della malattia in relazione ad effetti stagionali e misure di distanziamento sociale.\\
		I risultati dello studio mostra che il vero numero di casi probabilmente è molto più grande di quello effettivamente registrato, quasi il doppio degli infetti. Lo studio avverte sul fatto che i dati potrebbero dichiarare molti meno casi di quelli che sono nella realtà portando la popolazione e i governi a sottovalutare la situazione.
		\\
		\\
		Il prossimo documento di cui si parlerà è \emph{"How simulation modelling can help reduce the impact of COVID-19"} a cura della professoressa Christine Currie pubblicato dal \emph{CORMSIS}, centro di ricerca operativa e statistica dell'Università del Southamoton, il 15 Aprile 2020 su Taylor\&Francis Online il quale tratta dei vari modelli utilizzati dai governi e dall'Organizzazione Mondiale della Sanità per decidere le migliori strategie per debellare la minaccia del CoViD-19. I modelli utilizzati sono modelli epidemiologici volti a comprendere la diffusione della malattia. In quest'articolo verranno discussi di come i vari modelli di simulazione potrebbe aiutare a prendere decisioni.
		Il presente articolo presenta due obbiettivi, il primo è dare una guida ai diversi modelli di simulazioni e quali tipi possono essere utili a prendere decisioni durante un'emergenza sanitaria; il secondo è una \emph{"Chiamata alle armi"} dei modellatori di simulazioni per migliorare le ricerche.
		I modelli di tipo epidemiologico sono un'idea per predire il numero di nuovi casi identificando le migliori misure per ridurre i contagi, ma non aiutano ad organizzare l'organizzazione delle terapie intensive all'interno degli ospedali. Come nel paper precedente anche nel presente si indica il modello \emph{SEIR} come il modello più popolare per descrivere l'attuale pandemia per via del tempo di incubazione dato dallo stato di "Exposed".\\
		Questo paper considera quattro metodi di modellazione:
		\begin{itemize}
			\item \emph{Sistemi dinamici}: è un modello basato su equazioni differenziali che rappresentano azioni nel mondo.
			\item \emph{Modello basato ad agenti} può essere utilizzato per modellare l'interazione tra gli individui. In questi modelli utilizzati per monitorare la diffusione di una malattia è molto importante gestire le reti sociali e i movimenti nello spazio.
			\item \emph{Simulazione ad eventi discreti}, il sistema è rappresentato, nella sua evoluzione nel tempo, con variabili che cambiano istantaneamente il loro valore in ben definiti istanti di tempo appartenenti ad un insieme numerabili, questi istanti sono definiti eventi.
			\item \emph{Simulazione ibrida}: sono modelli che combinano due o più tecniche di modellazione citate precedentemente.
		\end{itemize}
	Dato che le simulazioni sviluppate con i metodi descritti in precedenza possono diventare molto complesse e richiedere tempo per l'esecuzione sono state sviluppate le simulazioni distribuite, le quali permettono di collegare più simulazioni in una singola in scala maggiore permettendo una velocità maggiore.\\ 
	Di seguito il presente paper introduce le tipologie di decisioni che sono state prese grazie all'utilizzo di simulazioni durante la pandemia da CoViD-19; esse vengono divise in tre sezioni:
	\begin{itemize}
		\item \emph{Decisioni che hanno avuto effetto sulla diffusione del virus.}
		\item \emph{Decisioni che riguardano la gestione delle risorse.}
		\item \emph{Decisioni sulla salute della popolazione.}
	\end{itemize}
	\myworries{Inserire immagini decisioni}\\
	Ora andiamo più nel dettaglio nelle decisioni per evitare la diffusione del virus descritte nel presente paper.\\
	Come prima decisione possibile abbiamo quella della \emph{quarantena}, questo per evitare del tutto la diffusione del virus. Questa decisione è d'obbligo una volta individuato un caso di corona virus, questo fino alla sua completa guarigione in modo tale da non poter contagiare nessuno. L'articolo cita un articolo del 2016 intitolato "The Ebola crisis and the corresponding public behavior: A system dynamics approach" il quale dimostra di come la messa in quarantena dei casi di ebola ha avuto un impatto importante su combattere la malattia; questo però non è applicabile così facilmente alla pandemia da corona virus dato che i sintomi dell'ebola sono ben visibili sulla pelle dei malcapitati, a differenza di quelli da CoViD-19 che possono anche essere lievi o assenti.\\
	Una seconda decisione che ha avuto un grande impatto sul nostro stile di vita, ma allo stesso è stato di grande aiuto per combattere la diffusione del virus sono le misure di \emph{distanziamento sociale}. Per distanziamento sociale si intendono quei comportamenti nel quale riduciamo, in parte ,il contatto umano in modo tale da essere sicuri di essere contagiati o di contagiare e quindi puntare a ridurre la crescita della curva dei contagi. Esempi di distanziamento sociale possono essere: \begin{itemize}
		\item mantenere una distanza di sicurezza di circa due metri in modo tale da non far arrivare gocce di saliva alla persona con cui stiamo colloquiando
		\item lavorare da casa quando è possibile, infatti molte aziende sono passate allo smart-working, in modo tale da continuare a lavorare tranquillamente anche da casa.
		\item Cancellare eventi i quali prevedono un elevato numero di persone.
		\item chiudere le scuole e optare per le lezioni in remoto.
		\item evitare di incontrare fisicamente amici e parenti.
	\end{itemize}
	Un decisione molto più drastica delle due precedenti è il lock-down imposto dal governo in modo tale da costringere la popolazione al distanziamento sociale e a lasciare le abitazioni sono in caso di prima necessità, come andare a fare la spesa, andare a lavorare ed uscire per motivi di salute. Per ovviare a problemi legati all'economia ogni governo può includere una soglia di valori sul quando rilassare le restrizioni in modo tale da permettere ai lavoratori che non eseguono un lavoro considerato di "Prima necessità" di lavorare facendo girare l'economia.\\
	In conclusione il documento lascia un appello definendolo "The call to arms" invitando a sviluppare simulazioni in merito alla diffusione e ai comportamenti per combattere la diffusione del virus in modo tale da supportare la ricerca, per poi fare l'esempio dei "codebreakers" britannici insieme ad Alan Turing, durante la Seconda Guerra Mondiale, decriptarono Enigma mostrando che l'unione di un buon team risolve problemi difficili.\\
	Lo sviluppo delle simulazioni si è dimostrato essere molto efficace, basti pensare al fatto che questo paper è stato redatto il 27 Marzo 2020 e seguendo le decisioni suggerite dalle simulazioni stiamo pian piano tornando alla normalità.\\
	
		
\end{document}