\documentclass {article}
\usepackage[utf8]{inputenc}
\usepackage [a4paper,left=1cm,bottom=1.5cm,right=1cm,top=1cm]{geometry}
\usepackage {graphicx}
\usepackage [svgnames]{xcolor}
\usepackage[nowrite,infront,standard,swapnames]{frontespizio}
\usepackage{listings}
\usepackage{pdfpages}
\nofiles 
\fontoptionnormal 
\renewcommand {\fronttitlefont }{\fontsize {16}{20} \bfseries }
\renewcommand {\frontinstitutionfont }{\fontsize {16}{20} \bfseries }
\renewcommand {\frontdivisionfont }{\fontsize {14}{18} \bfseries }
\renewcommand {\frontpretitlefont }{\fontsize {14}{18} \bfseries }
\Universita {Salerno}
\Dipartimento {Informatica}
\Corso [Laurea]{Informatica}
\Annoaccademico {2020-2021}
\Titolo {\sf Simulazione 3D ad Agenti Riguardo Contagi su Mezzo di Trasporto Pubblico}
\Sottotitolo{\sf }
\Logo [4cm]{immagini/logounisa}
\NCandidato{Candidato}
\Candidato [0512105806]{Antonio Renato Montefusco}
\Relatore {Ch.mo Prof.\ Vittorio Scarano}
\Margini {3cm}{3cm}{3cm}{3cm} 


\begin{document}
	
	\preparefrontpagestandard
	\Large{Abstract\\ Data la pandemia da Covid-19, si sono dovuti effettuare vari studi e ricerche sull'argomento in ogni ambito. Lo scopo di questa tesi è la realizzazione di una simulazione basata su agenti che utilizzano un mezzo di trasporto pubblico in periodo di pandemia. Tramite questa simulazione è possibile ottenere dati statistici sui contagi che avvengono in una giornata sulla tratta del mezzo di trasporto. La simulazione è stata realizzata completamente in Unity 3D. Gli agenti utilizzati per la simulazione sono dei modelli di persona e sono animati per tutte le loro azioni, che siano la camminata, il sedersi sul bus o lo scendere da quest'ultimo. Per la simulazione sono stati utilizzati dati reali ottenuti da un articolo scientifico riguardante i trasporti nella città giapponese di Obuse pubblicato da Elsevier LTD nel 2020, il quale studio ha riportato il numero di passeggeri medi durante la tratta di autobus tracciati tramite Wi-Fi scanner.
}
	
\end{document}